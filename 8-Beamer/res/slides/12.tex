\subsection{Transizioni} 
\begin{frame}
  \frametitle{Transizioni nei cambi slide}
  
  Anche se in pdf, \LaTeX{} permette effetti sulle slide
  \begin{itemize}
    \item Abbelliscono la presentazione
    \item La rendono meno ``pdf'' e più ``powerpoint''
    \item Possibilità di impostare un timer automatico per l'avanzamento 
automatico delle slide
  \end{itemize}

\end{frame}

\begin{frame}[allowframebreaks]
  \frametitle{Tabella transizioni}
  
  Tabella degli effetti disponibili
 
% FIXME: sta tabella è sbarellatissima, da mettere apposto perché il testo 
% sborda

\tablefirsthead{\hline \textbf{Keyword}       & \textbf{Effetto}           \\}
\tablehead{\hline \textbf{Keyword}       & \textbf{Effetto}           \\}

\begin{table}[t]
\centering
\begin{xtabular}{|c|p{5cm}|}
\hline

\texttt{\textbackslash transblindshorizontal} & La slide successiva scorre 
sopra quella precedente con un movimento da sinistra verso destra \\ \hline
\texttt{\textbackslash transblindsvertical}   & La slide successiva scorre 
sopra quella precedente con un movimento dall'alto verso il basso \\ \hline
\texttt{\textbackslash transboxin}            & La slide precedente viene 
``assorbita'' in un quadrato da quella nuova                       \\ \hline
\texttt{\textbackslash transboxout}           & La slide successiva si espande 
a forma di quadrato su quella precedente                      \\ \hline
\texttt{\textbackslash transdissolve}         & La slide successiva si dissolve 
in un effetto ``pixelato''                                   \\ \hline
\texttt{\textbackslash transglitter}          & Un misto tra una dissolvenza a 
pixel e una transizione orizzontale                           \\ \hline
\texttt{\textbackslash transwipe}             & La slide successiva esegue uno 
``swipe'' sopra quella precedente da sinistra verso destra    \\ \hline
\texttt{\textbackslash transfade}             & Effetto di fade tra la slide 
precedente e quella successiva                                  \\ \hline
\end{xtabular}
\caption[Tabella degli effetti Beamer]{Tabella degli effetti per Beamer. Nota:
solo uno di questi effetti può essere applicato per pagina: l'applicazione di
due o più effetti nella stessa slide causerà un errore durante la compilazione.}
\label{tab:beamer_effects}
\end{table}
\end{frame}
