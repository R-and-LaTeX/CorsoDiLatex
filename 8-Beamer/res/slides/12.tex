\subsection{Transizioni} 
\begin{frame}
  \frametitle{Transizioni nei cambi slide}
  
  Anche se in pdf, \LaTeX{} permette effetti sulle slide
  \begin{itemize}
    \item Abbelliscono la presentazione
    \item La rendono meno ``pdf'' e più ``powerpoint''
    \item Possibilità di impostare un timer automatico per l'avanzamento 
automatico delle slide
  \end{itemize}

\end{frame}

\begin{frame}
  \frametitle{Transizioni nei cambi slide - 2}
  
  Tabella degli effetti disponibili
 
% FIXME: sta tabella è sbarellatissima, da mettere apposto perché il testo 
% sborda
\begin{table}[t]
\centering
\begin{tabular}{|l|l|}
\hline
\textbf{Keyword}       & \textbf{Effetto}           \\ \hline
transblindshorizontal & La slide successiva scorre sopra quella
precedente con un movimento da sinistra verso destra \\ \hline
transblindsvertical   & La slide successiva scorre sopra quella
precedente con un movimento dall'alto verso il basso \\ \hline
transboxin            & La slide precedente viene ``assorbita'' in un
quadrato da quella nuova                       \\ \hline
transboxout           & La slide successiva si espande a forma di
quadrato su quella precedente                      \\ \hline
transdissolve         & La slide successiva si dissolve in un effetto
``pixelato''                                   \\ \hline
transglitter          & Un misto tra una dissolvenza a pixel e una
transizione orizzontale                           \\ \hline
transwipe             & La slide successiva esegue uno ``swipe'' sopra
quella precedente da sinistra verso destra    \\ \hline
transfade             & Effetto di fade tra la slide precedente e
quella successiva                                  \\ \hline
\end{tabular}
\caption[Tabella degli effetti Beamer]{Tabella degli effetti per Beamer. Nota:
solo uno di questi effetti può essere applicato per pagina: l'applicazione di
due o più effetti nella stessa slide causerà un errore durante la compilazione.}
\label{tab:beamer_effects}
\end{table}
\end{frame}
