\section{Linux e MacOS}
\subsection{Installazione}
\begin{frame}
  \frametitle{Compilazione su altri sistemi operativi}
  
  Per GNU/Linux
  \begin{itemize}
   \item Installare \href{http://www.tug.org/texlive/}{TexLive} col proprio 
gestore di pacchetti (\texttt{apt}, \texttt{pacman}, ...)
   \item Usare un IDE per Linux
   \begin{itemize}
    \item Kile
    \item TeXStudio
    \item Emacs (editor di file generico)
    \item Sublime (editor di file generico)
   \end{itemize}
  \end{itemize}
  
  Per MacOS
  \begin{itemize}
   \item Installare \href{http://www.tug.org/mactex/}{MacTeX} e assicurarsi che 
\texttt{latexmk} sia presente nella vostra installazione 
    % Non ho un Mac, non ho modo di provare a vedere se c'è :(
  \end{itemize}
\end{frame}


\subsection{Compilazione da terminale}
\begin{frame}
 \frametitle{Compilazione da terminale}
 
 \begin{itemize}
  \item Più comoda e veloce
  \item \texttt{latexmk} semplifica la compilazione
 \end{itemize}
 
 % FIXME: vogliamo metterci un esempio anche? La slide è vuota praticamente

\end{frame}
