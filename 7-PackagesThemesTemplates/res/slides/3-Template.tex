\begin{frame}{Template}

\begin{itemize}
\item I \emph{template} invece possiamo pensarli delle raccolte di comandi che
definiscono l'aspetto e i comandi di per un certo documento
\item Possono definire anche i pacchetti utilizzati
\item Non definiremo mai un template ma ne sono disponibili molti online
\item Vediamo un esempio di template per la tesi!
\end{itemize}

\begin{figure}
	\centering
	\includegraphics[scale=0.20]{res/images/template}
\end{figure}

\end{frame}