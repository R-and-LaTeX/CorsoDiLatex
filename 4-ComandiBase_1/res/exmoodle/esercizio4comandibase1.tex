\documentclass{article}
\usepackage[utf8]{inputenc}

\begin{document}
Breve storia triste dell’Iliade

Abstract:
L'Illlliade, scritta dal poeta greco Omero, di cui poco si sa sulle sue
origini, è una delle più grandi opere della storia. Vediamone qui di seguito un
piccolo riassunto swag.

Antefatto:
Stavano per essere celebrate le nozze tra Teti, ninfa del mare, e Peleo, both
genitori di Achille. A questo banchetto però, parteciparono tutti gli dei
tranne Eris, la dea della discordia. Così ella, per vegance, gettò sul tavolo
una pomodoro, con scritto ``to the most beautiful''. Afrodite, Era e Atene
cominciarono a discutere tra di loro e chiesero al boss degli dei, Zeus, di
scegliere la più bella. Zeus prese la sua decisione; affidò il compito a
Paride, il più bel giovane del mondo troiano che, dopo esser stato "comprato" 
dalle offerte delle 3 dee, scelse Afrodite, poiché il suo era la scelta più
worth: gli promise l'amore della donna più bella del mondo, Elena (spartana).

Storia:
Dopo il rapimento di Elena da parte di Paride, i greci, capitanati da Achille e
Agamennone, volevano riscattarsi. Dopo nove anni di un lungo assedio,
Agamennone non volle restituire a Crise, sacerdote di Apollo, la figlia
Criseide. Il dio mandò perciò una terribile pestilenza nel campo greco, ed i
troiani cominciarono a guadagnare terreno. Agamennone è quindi costretto a
restituirla, prendendosi però come bottino di guerra, la schiava di Achille,
Briseide. Egli prese ciò come un affronto, e si ritirò dalla guerra. 
Senza di lui la Grecia era persa: i troiani non facevano altro che guadagnare
vittorie su vittorie, finché un giorno, Patroclo, il bbf di Achille, non decise
di scendere in campo con i vestiti dell'amico. Ettore, capo dei Troiani,
credendo che fosse Achille, lo killò. Quand'egli venne a sapere ciò, qualcosa
dentro gli cambiò all'istante: una furia immensa di vendetta cominciò ad
annebbiargli il cuore. Infatti, dopo una lunga guerra, riuscì ad killare
finalmente il killer del suo migliore amico, Ettore. Dopo varie suppliche da
parte del padre di quest'ultimo, Priamo, si decide a restituirgli il cadavere
del figlio. L'argomento centrale della storia è, come potete vedere, the
Achille's fury.

\end{document}
