\documentclass{article}
\usepackage[utf8]{inputenc}
\usepackage{ulem}

\begin{document}
\textbf{Breve storia triste dell’Iliade}

\textbf{Abstract}:
L'\sout{Illlliade}, scritta dal poeta greco \underline{Omero}, di cui poco si
sa sulle sue origini, è una delle più grandi opere della storia. Vediamone qui
di seguito un piccolo riassunto \sout{\textit{swag}}.

\textbf{Antefatto}:
Stavano per essere celebrate le nozze tra \underline{Teti}, ninfa del mare, e 
\underline{Peleo}, 
\textit{both} genitori di \underline{Achille}. A questo banchetto però,
parteciparono tutti gli dei tranne \underline{Eris}, la dea della discordia.
Così ella, per \textit{vegance}, gettò sul tavolo \sout{una pomodoro}, con
scritto ``\textit{to the most beautiful}''. \underline{Afrodite}, 
\underline{Era}
e \underline{Atena} cominciarono a discutere tra di loro e chiesero al 
\textit{boss} degli dei, \underline{Zeus}, di scegliere la più bella.
\underline{Zeus} prese la sua decisione; affidò il compito a 
\underline{Paride}, il più bel giovane del mondo troiano che, dopo esser stato
"comprato" dalle offerte delle 3 dee, scelse \underline{Afrodite}, poiché il
suo era la scelta più \textit{worth}: gli promise l'amore della donna più bella
del mondo, \underline{Elena} (spartana).

\textbf{Storia}:
Dopo il rapimento di \underline{Elena} da parte di \underline{Paride}, i greci,
capitanati da \underline{Achille} e \underline{Agamennone}, volevano
riscattarsi. Dopo nove anni di un lungo assedio, \underline{Agamennone} non
volle restituire a \underline{Crise}, sacerdote di \underline{Apollo}, la figlia
\underline{Criseide}. Il dio mandò perciò una terribile pestilenza nel campo
greco, ed i troiani cominciarono a guadagnare terreno. \underline{Agamennone} è
quindi costretto a restituirla, prendendosi però come bottino di guerra, la
schiava di \underline{Achille},
\underline{Briseide}. Egli prese ciò come un affronto, e si ritirò dalla
guerra. Senza di lui la Grecia era persa: i troiani non facevano altro che
guadagnare vittorie su vittorie, finché un giorno, \underline{Patroclo}, il 
\textit{bbf} di \underline{Achille}, non decise di scendere in campo con i
vestiti dell'amico. \underline{Ettore}, capo dei Troiani, credendo che fosse 
\underline{Achille}, lo \sout{killò}. Quand'egli venne a sapere ciò, qualcosa
dentro gli cambiò all'istante: una furia immensa di vendetta cominciò ad
annebbiargli il cuore. Infatti, dopo una lunga guerra, riuscì ad \sout{killare}
finalmente il \textit{killer} del suo migliore amico, \underline{Ettore}. Dopo
varie suppliche da parte del padre di quest'ultimo, \underline{Priamo}, si
decide a restituirgli il cadavere del figlio. L'argomento centrale della storia
è, come potete vedere, \textit{the \underline{Achille}'s fury}.

\end{document}
