\begin{frame}
 \frametitle{Personalizzazione}
 
 Gli elenchi possono essere personalizzati a piacimento!
 \begin{itemize}
  \item[-]<1-> possiamo inserire
  \item[/]<2-> il simbolo
  \item[.]<3-> che più ci piace!
 \end{itemize}

\end{frame}

\begin{frame}
 
 \begin{center}
  \huge Come è possibile questa stregoneria?!
 \end{center}
 
 \begin{figure}[h]
  \centering
  \includegraphics{witchcraft}
 \end{figure}

 
  
%  \begin{textblock*}{5cm}(4cm,5cm)
%    \includegraphics[scale=0.75]{witchcraft}
%  \end{textblock*}

\end{frame}

\begin{frame}[fragile]
 \frametitle{Personalizzazione - 2}
 
 Il segreto sta nell'usare le parentesi quadre dopo \texttt{\textbackslash item}
 
 \begin{exampleblock}{Esempio}
 \begin{lstlisting}[frame = single, title={Una lista puntata personalizzata}] 
\begin{itemize}
 \item[-] Tizio
 \item[/] Caio
 \item[.] Sempronio
\end{itemize}
  \end{lstlisting}
 \end{exampleblock}
 
 All'interno possiamo metterci anche parole! Ma attenzione...
\end{frame}

\begin{frame}
 \frametitle{Personalizzazione - 3}
 
 \begin{itemize}
  % Intenzionale
  \item[parole troppo lunghe] potrebbero rompere la formattazione!
 \end{itemize}

\end{frame}
