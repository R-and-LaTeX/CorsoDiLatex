\begin{frame}{Esercizi 3 e 4}

\begin{block}{Esercizio 3 - Ricreate questo output}
Da qui sono nati certi modi di dire \sout{entrati nel gergo} dell'
\textbf{autostoppista}, come ad esempio nella frase: \textit{Ehi, 
\underline{ciacci} quel \underline{ganzo} di Ford Prefect? È un
\underline{frisco} che sa davvero dove ci ha l’asciugamano!} (\textbf{Ciacciare}
 = conoscere, rendersi conto di, incontrare, avere rapporti sessuali con; 
\textbf{ganzo} = tipo proprio in gamba; \textbf{frisco} = tipo
straordinariamente in gamba).
\end{block}

\begin{block}{Esercizio 4}
	Prendendo il testo a questo url \url{http://www.google.it} mettete
	\begin{itemize}
	\item I titoli in grassetto
	\item Le parole in inglese in corsivo
	\item Le parti sbagliate sbarrate
	\item I nomi di persona sottolineati
	\end{itemize}
\end{block}

\end{frame}