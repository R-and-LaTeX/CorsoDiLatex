\documentclass{beamer}
\usepackage[utf8]{inputenc}
\usepackage{listings}
\lstset{
    language=[LaTeX]Tex,%C++,
    keywordstyle=\color{blue}, %\bfseries,
    basicstyle=\small\ttfamily,
    commentstyle=\color{green}\ttfamily,
    stringstyle=\rmfamily,
    numbers=left,%
    numberstyle=\scriptsize, %\tiny
    stepnumber=1,
    numbersep=8pt,
    showstringspaces=false,
    breaklines=true,
    frameround=ftff,
    frame=single
}
\usepackage{../UnipdTheme/Padova/beamerthemePadova}
\usepackage[absolute,overlay]{textpos}

\title{Bibliografia e creazione dei comandi}
\subtitle{Solo per stomaci forti}
\author{Davide Polonio \& Marco Zanella}
\date{January 1, 1900}


\begin{document}

	\maketitle
	\include{res/slides/1-Bibliografia}
	\begin{frame}{Bibliografia manuale}

\begin{textblock*}{2cm}(10cm,1.5cm)
      \includegraphics[scale=0.30]{res/images/manual}
\end{textblock*}

È la scelta più semplice ma presenta alcune limitazioni

\vfill

Comandi da utilizzare:
\begin{itemize}
	\item \texttt{\textbackslash{}begin\{thebibliography\}\{n\}\dots{}
	\textbackslash{}end\{thebibliography\}} ambiente in cui dichiarare le opere
	\item \texttt{\textbackslash{}bibitem[etichetta\_personalizzate]
	\{identificativo\}} dove
	\begin{itemize}
		\item \textcolor{red}{\texttt{etichetta\_personalizzate}} è opzionale
		e permette di definire etichette personalizzate
		\item \textcolor{red}{\texttt{identificativo}} è una label e si
		consiglia la sintassi \texttt{autore:titolo}
	\end{itemize}
	\item \texttt{\textbackslash{}cite\{identificativo\}} per inserire una
	citazione
\end{itemize}

\end{frame}
    \begin{frame}[fragile]{Inserimento bibliografia manuale}

\begin{textblock*}{2cm}(10cm,1.5cm)
      \includegraphics[scale=0.30]{res/images/manual}
\end{textblock*}

Nel caso di \texttt{book} o \texttt{report}
\begin{lstlisting}
\cleardoublepage
%\phantomsection
\addcontentsline{toc}{chapter}{\bibname}
\end{lstlisting}

\vfill

Nel caso di \texttt{article}
\begin{lstlisting}
\clearpage
%\phantomsection
\addcontentsline{toc}{section}{\refname}
\end{lstlisting}

\end{frame}
    \begin{frame}[fragile]{Esempio bibliografia manuale - 1}

\begin{exampleblock}{Bibliografia manuale - Preambolo, testo e inserimento}
	\begin{lstlisting}
\documentclass{book}

\begin{document}

\dots{}citazione di un'opera~\cite{autore1:titolo1} bla bla bla\dots{}
\\
\dots{}citazione di un'opera~\cite{autore2:titolo2} bla bla bla\dots{}

\cleardoublepage
%\phantomsection
\addcontentsline{toc}{chapter}{\bibname}
	\end{lstlisting}
\end{exampleblock}

\end{frame}
    \begin{frame}[fragile]{Esempio bibliografia manuale - 2}

\begin{exampleblock} {Bibliografia manuale - codice bibliografia}
	\begin{lstlisting}
\begin{thebibliography}{9}

\bibitem{autore1:titolo1} Cognome, Nome(1979), \emph{Titolo completo}.
\bibitem[etichetta\_personalizzata]{autore2:titolo2} Cognome, Nome(1979),
\emph {Titolo completo}.

\end{thebibliography}

\end{document}
	\end{lstlisting}
\end{exampleblock}

\end{frame}
    \include{res/slides/6-ProEControBiBliografiaManuale}
    \begin{frame}{Bibliografia automatica}

\begin{textblock*}{2cm}(10cm,2cm)
      \includegraphics[scale=0.28]{res/images/automatic}
\end{textblock*}

È più complessa ma permette un grado di personalizzazione maggiore

\vfill

Per utilizzarla è necessario
\begin{itemize}
	\item Utilizzare il pacchetto \texttt{biblatex} al quale è possibile è possibile specificare una serie di opzioni
	\begin{itemize}
		\item \texttt{backend=biber}
		\item \texttt{style} per definire lo stile delle citazioni
		\item \texttt{hyperref} e \texttt{backref} nel caso in cui si voglia
		che ci siano nel documento i collegamenti tra citazione e dati
		bibliografici
	\end{itemize}
	\item importare il pacchetto per le citazioni con il comando 
	\texttt{\textbackslash{}usepackage[autostyle,italian=guillemets]
	\{csquotes\}}
	\item È necessario creare un \emph{database} delle opere
\end{itemize}

\end{frame}
    \include{res/slides/8-StiliCitazione}
    \begin{frame}[fragile]{Bibliografia automatica - Database opere}

\begin{exampleblock}{Esempio opera}
	\begin{lstlisting}
@book{eco:tesi,
  author     = {Eco, Umberto},
  title      = {Come si fa una tesi di laurea},
  publisher  = {Bompiani},
  date       = {1977},
  location   = {Milano},
}
	\end{lstlisting}
\end{exampleblock}

\end{frame}
    \begin{frame}[fragile]{Inserimento bibliografia automatica}

\begin{lstlisting}
%vedi bibliografia manuale per il significato
\cleardoublepage
%per inserire anche i riferimenti senza citazioni
\nocite{*}
%per stampare effettivamente la bibliografia
\printbibliography
\end{lstlisting}

\end{frame}
    \begin{frame}[fragile]{Esempio bibliografia automatica - 1}

\begin{exampleblock}{Bibliografia automatica - Preambolo, testo e inserimento}
	\begin{lstlisting}
\documentclass{book}
\usepackage[utf8]{inputenc}
\usepackage{hyperref}
\usepackage[autostyle, italian=guillemets]{csquotes}
\usepackage[backend=biber, style=alphabetic, hyperref, backref]{biblatex}
\addbibresource{bibliografia.bib}
\begin{document}
\dots{}citazione di un'opera~\cite{eco:tesi} bla bla bla\dots{}\\
\dots{}citazione di un'opera~\cite{mori:tesi} bla bla bla\dots{}
\cleardoublepage \nocite {*}
\printbibliography
\end{document}
	\end{lstlisting}
\end{exampleblock}

\end{frame}
    \begin{frame}[fragile]{Esempio bibliografia automatica - 2}

\begin{exampleblock} {File bibliografia.bib}
	\begin{lstlisting}
@book{eco:tesi,
  author       = {Eco, Umberto},
  title        = {Come si fa una tesi di laurea},
  publisher    = {Bompiani},
  date         = {1977},
  location     = {Milano},
}
@article{mori:tesi,
  author       = {Mori, Lapo Filippo},
  title        = {Scrivere la tesi di laurea con \LaTeXe},
  journaltitle = {Giornale},
  number       = {3},
  date         = {2007},
}
	\end{lstlisting}
\end{exampleblock}

\end{frame}
    \begin{frame}[fragile]{Creazione comandi personalizzati}

Spesso può risultare utile creare un nuovo comando
\begin{itemize}
	\item Per eseguire ``azioni'' ripetitive
	\item In modo tale da modificare in modo semplice più parti
\end{itemize}

La sintassi per definire un nuovo comando è
\begin{center}
    \texttt{\textbackslash{}newcommand\{nome\_comando\}[argomenti]
    \{definizione\}}
\end{center}

\begin{exampleblock}{Esempio nuovo comando}
	\begin{lstlisting}
\newcommand{\trecolori}[3]{
    \textcolor{red}{#1} 
    \textcolor{green}{#2} 
    \textcolor{blue}{#3}
}
	\end{lstlisting}
\end{exampleblock}


\end{frame}


\end{document}