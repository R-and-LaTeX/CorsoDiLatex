\begin{frame}{Bibliografia automatica}

\begin{textblock*}{2cm}(10cm,2cm)
      \includegraphics[scale=0.28]{res/images/automatic}
\end{textblock*}

È più complessa ma permette un grado di personalizzazione maggiore

\vfill

Per utilizzarla è necessario
\begin{itemize}
	\item Utilizzare il pacchetto \texttt{biblatex} al quale è possibile è possibile specificare una serie di opzioni
	\begin{itemize}
		\item \texttt{backend=biber}
		\item \texttt{style} per definire lo stile delle citazioni
		\item \texttt{hyperref} e \texttt{backref} nel caso in cui si voglia
		che ci siano nel documento i collegamenti tra citazione e dati
		bibliografici
	\end{itemize}
	\item importare il pacchetto per le citazioni con il comando 
	\texttt{\textbackslash{}usepackage[autostyle,italian=guillemets]
	\{csquotes\}}
	\item È necessario creare un \emph{database} delle opere
\end{itemize}

\end{frame}