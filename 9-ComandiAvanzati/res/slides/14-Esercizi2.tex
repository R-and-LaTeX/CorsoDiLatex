\begin{frame}[fragile]{Esercizi comandi personalizzati}

\begin{block}{Esercizio 2 - Wikipedia command}
    
Definire un nuovo comando \texttt{\textbackslash wiki} che completi un url di wikipedia.

Il comando deve accettare due parametri; il primo è il testo da visualizzare al posto dell'url ed è opzionale, il secondo è la sezione di testo finale dell'url, cioè quella che ne contraddistingue il contenuto.

\end{block}

\begin{block}{Esempi}

    \begin{itemize}
        \item \texttt{\textbackslash myWiki\{Latex\}} = \myWiki{Latex}
        \item \texttt{\textbackslash myWiki[l'autore di Latex]\{Leslie\_Lamport\}} = \myWiki[l'autore di Latex]{Leslie_Lamport}
    \end{itemize}

\end{block}

\begin{figure}[l]
    \includegraphics[scale=0.03, left]{res/images/wiki.png}
\end{figure}

\end{frame}
