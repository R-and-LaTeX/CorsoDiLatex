\begin{frame}[fragile]
 
  \frametitle{Altri parametri e comandi}

  Parametri e comandi ``avanzati''
  
  \begin{itemize}
    \item Il parametro \texttt{@\{\dots\}} specifica come separare le colonne (es.
\texttt{@\{\}} e \texttt{@\{,\}}).
    \item Il comando \texttt{\textbackslash hline\{X-Y\}} inserisce una linea
orizzontale che va dalla colonna X alla colonna Y.
    \item Il comano \texttt{\textbackslash multicolumn\{numero\_colonne\}
\{allineamento\}\{testo\}} crea una cella multicolonna (risp. \texttt{\textbackslash
multirow\{numero\_righe\}\{larghezza\}\{testo\}} per una cella multiriga)
  \end{itemize}

\end{frame}

\begin{frame}
 
 \frametitle{Esempi 1/2}
  
  \begin{centering}

  \begin{tabular}{|p{1,5cm}|l|p{2cm}|}
   \hline
   a & aa & aaa\\
   \hline
  \end{tabular}
  \vspace{1cm}

  \begin{tabular}{|c|c|c|c|}
   \hline
   a & aa & aaa & aaaa\\
   \cline{2-4}
   b & bb & bbb & bbbb\\
   \cline{3-4}
   c & cc & ccc & cccc\\
   \hline
  \end{tabular}
  \vspace{1cm}

  \begin{tabular}{l@{,}l@{}}
   123 & 45 + \\
   678 & 90 = \\
   \hline
   802 & 35
  \end{tabular}

  \end{centering}

\end{frame}
