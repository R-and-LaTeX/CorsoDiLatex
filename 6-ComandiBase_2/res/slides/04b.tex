\begin{frame}
  \frametitle{Tabelle Semplici - 3}
  \vspace{1,7cm}
  \begin{itemize}
    \item<1-> Iniziamo un nuovo blocco, \texttt{tabular} in cui specifichiamo
il numero di colonne e definiamo l'allineamento del loro contenuto:
    \begin{itemize}
      \item<2-> \texttt{\textcolor{red}{l}} - left
      \item<3-> \texttt{\textcolor{red}{c}} - center
      \item<4-> \texttt{\textcolor{red}{r}} - right
    \end{itemize}
   \item<5-> Si può definire la larghezza delle colonne con
\texttt{p\{larghezza\}}.
   \item<6-> In ogni riga il comando \& separa le colonne; ogni riga termina
con \textbackslash \textbackslash
   \item<7-> \texttt{\textbackslash caption} permette di inserire una 
didascalia nella tabella      
  \end{itemize}

  \begin{textblock*}{5cm}(8.2cm,1.2cm)
   \includegraphics[scale=0.08]{lens}
  \end{textblock*}

\end{frame}
